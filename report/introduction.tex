\section{Introduction}
Monocular vision sensors are probably the most widely used devices in embedded systems, where the payload is a serious parameter to take into consideration, even more so for flying robots. Depending on the application, whether it be obstacle avoidance or egomotion, several such devices may be used on the embedded system. 

Previous work from A. Briod \cite{ekf}, show that the egomotion estimates greatly benefit (statistically) from the addition of numerous pinhole cameras on the body of the robot. Nevertheless, this method entails a calibration procedure to estimate the viewing direction of each sensor with respect to the body of the robot, for which an automatic estimation thereof was proposed in \cite{autocalib}. Hence, another solution is to consider using omnidirectional cameras. 

An omnidirectional camera provides a wide field of view of at least 180 degrees. In our case of interest, the dioptric camera that was used consists in a combination of shaped lenses (fisheye lenses) and typically can reach a field of view slightly larger than 180 degrees. Thanks to the calibration of a model of the camera, mapping each camera image pixels to the 3D scene, these devices enable the sampling of many points on the camera image without the need to re-estimate the corresponding viewing directions each time they are changed. More importantly, a wide field-of-view is preferred for optic-flow-based egomotion estimation because it is necessary to distinguish clearly the direction of motion out of multiple optic-flow measurements. It thus allows for more robustness and redundancy in a variety of environments and conditions. 

The optical flow (or optic-flow) is the apparent motion of objects, surfaces and edges in a visual scene and can be obtained can be obtained either from the variation of pixel intensity (Lucas-Kanade method) or pattern displacement. This approach has two advantages compared to feature-based techniques. (1)~The optic-flow can be estimated from low-resolution sensors, which are typically small and lightweight, allowing for a low computational overhead. (2)~The optic-flow measurements can be measured from any scene whether recognizable features are present or not, which enables its use in both indoor and outdoor environments.

But the use of optic-flow measurements has several drawbacks compared to feature tracking when it comes to ego-motion estimation. (1)~The optic-flow is caused by the relative motion between the camera and the scene from which some visual cues are extracted. This makes the measurements particularly sensible to outliers induced by large uniform textures (e.g. white walls) or moving objects in the environment. (2)~The visual cues which are used for optic-flow computation vary at each timestep causing the distance to the scene to change. Thus, the estimation of the correct scaling of the velocity of the drone may be difficult and typically requires additional sensors such as inertial measurement units \cite{ekf} or sonar sensors.

Several approches were proposed for inferring the direction of motion from optic-flow measurements. A first approach is to rely on coarse epipolar constraints to estimate the position of the focus of expansion, which is the single point from which the optic-flow field diverges. In this case, the sampling of a high number of optic-flow measurements at each timestep and the computation of their corresponding epipolar constraints enable the refinement of the estimated position of the FOE \cite{lim}. Another approach would be to proceed to optic-flow and inertial sensors fusion \cite{ekf}. Relying on translational optic-flow direction constraint and derotation \cite{derotation}, this method enable the estimation of the direction as well as the scaling of the velocity.

However, the above methods exhibit some disadvantages. (1) The computation of epipolar constraints from raw optic-flow measurements is limited to antipodal points, thus reducing the interest in wide field-of-view cameras. (2) The estimation of the scaling factor using extended kalman filters (EKF) is not necessary in our particular case and requires inertial measurements units which are not present on the camera board. This method would thus entail the transmission of the pre-processed data from the camera chip to the autopilot. Hence, it would be necessary to estimate the additional delay due to preprocessing and transmission. In this case, the computation of optical flow measurements and viewing direction, the projection on the unit sphere, and the derotation, could be performed on the camera chip while the remaining computations (EKF predictions and updates) would be achieved by the autopilot.

In this report, we propose a method for estimating the direction of motion based on high frequency and unscaled optic-flow measurements. We characterize this method by means of a wide field-of-view camera on which a hundred of optic-flow measurements are computed and a motion capture system. Section II describes the theoretical background and the actual algorithm implemented on the camera board. Section III presents the experimental setup used for characterization and details the results of the experiments conducted in a synthetic environment.