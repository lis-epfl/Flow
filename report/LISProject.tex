\section{IMU}
An inertial measurement unit (IMU) is a device that uses gyroscopes and accelerometers to estimate the relative position, velocity, and acceleration of a moving vehicle. An IMU is also known as an Inertial Navigation System (INS), and it has become a common navigational component of aircraft and ships. 
An IMU estimates the six-degree-of-freedom (DOF) pose of the vehicle: position (x, y, z) and orientation (roll, pitch, yaw). Nevertheless, heading sensors like compasses and gyroscopes, which conversely only estimate orientation, are often improperly called IMUs.
Besides the 6-DOF pose of the vehicle, commercial IMUs also usually estimate velocity and acceleration. To estimate the velocity, the initial speed of the vehicle needs to be known. The working principle of an IMU is shown in figure 4.10. Let us suppose that our IMU has three orthogonal accelerometers and three orthogonal gyroscopes. The gyroscope data is integrated to estimate the vehicle orientation while the three accelerometers are used to estimate the instantaneous acceleration of the vehicle. The acceleration is then transformed to the local navigation frame by means of the current estimate of the vehicle orientation relative to gravity. At this point the gravity vector can be subtracted from the measurement. The resulting acceleration is then integrated to  obtain the velocity and then integrated again to obtain the position, provided that both the initial velocity and position are a priori known. To overcome the need of knowing of the initial velocity, the integration is typically started at rest (i.e., velocity equal to zero).
Observe that IMUs are extremely sensitive to measurement errors in both gyroscopes and accelerometers. For example, drift in the gyroscope unavoidably undermines the estimation of the vehicle orientation relative to gravity, which results in incorrect cancellation of the gravity vector. Additionally observe that, because the accelerometer data is integrated twice to obtain the position, any residual gravity vector results in a quadratic error in position. Because of this and the fact that any other error is integrated over time, drift is a fundamental problem in IMUs. After long period of operation, all IMUs drift. To cancel this drift, some reference to some external measurement is required. In many robot applications, this has been done using cameras or GPS. In particular, cameras allow the user to annihilate the drift every time a given feature of the environment—whose 3D position in the camera reference frame is known—is reobserved (see sections 4.2.6 or 5.8.5). Similarly, as described in the next section, GPS allows the user to correct the pose estimate every time the GPS signal is received.

\section{Omnidirectional cameras}
They are systems that cannot be described using conventional pinhole model because of the very high distortion by the imaging device.
An omnidirectional camera provides wide field of view, at least more than 180 degrees. There are several ways to build an omnidirectiona camera. One of them is the dioptric camera, which uses a combination of shaped lenses (e.g. fisheye lenses) and typically can reach a field of view slightly larger than 180 degrees.

\subsection{Central omnidirectional cameras}
A vision system is said to be central when the optical rays to the viewed objects intersect in a single point in 3D called projection center or single effective viewpoint. This property is called single effective viewpoint property.
All modern fisheye cameras are central, and hence, they satisfy the single effective viewpoint property. The reason a single effective viewpoint is so desirable is that it allows us to generate geometrically correct perspective images from the pictures captured by the omnidirectional camera. This is possible because, under the single view point constraint, every pixel in the sensed image measures the irradiance of the light passing through the viewpoint in one particular direction. When the geometry of the omnidirectional camera is known, that is, when the camera is calibrated, one can precompute this direction for each pixel. Therefore, the irradiance value measured by each pixel can be mapped onto a plane at any distance from the viewpoint to form a planar perspective image. Additionally, the image can be mapped on to a sphere centered on the single viewpoint, that is, spherical projection.
Another reason why the single viewpoint property is so important is that it allows us to apply the well known theory of epipolar geometry, which easily allows us to perform structure from motion. This theory holds for any central camera. Therefore, in those sections we will not make any distinction about the camera.

\subsection{Omnidirectional Camera Model and Calibration}
Intuitively, the model of an omnidirectional camera is a little more complicated than a standard perspective camera. The model should indeed take into account the refraction caused by the lens in the case of a fisheye camera. Because the literature in this field is quite large, here we review two different projectin models that have become standards in omnidirectional vision and robotics. The Taylor model, unifies both central catadioptric cameras and fisheye cameras under a general model.

\subsubsection{Unified Model for Catadioptric and Fisheye Cameras}
The unified model was proposed by Scaramuzza et al. in 2006. The main difference with the standard model lies in the choice of the function $g$. To overcome the lack of knowledge of a parametric model for fisheye cameras, a Taylor polynomial, whose coefficients and degree are found through the calibration process, is used.